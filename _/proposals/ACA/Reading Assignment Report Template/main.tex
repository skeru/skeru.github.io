\documentclass{article}
% <- this is a comment!
\usepackage{cite}

% Image inclusion
\usepackage{graphicx}
\graphicspath{{./fig/}} % Default path for figures
\DeclareGraphicsExtensions{.pdf,.jpeg,.png}

% Links
\usepackage[usenames,dvipsnames]{xcolor}
\usepackage[bookmarks, colorlinks, breaklinks]{hyperref}  
\hypersetup{linkcolor=blue,citecolor=blue,filecolor=black,urlcolor=MidnightBlue} 
% Tables
\usepackage{booktabs}

% Symbols
\usepackage{amsmath}
\usepackage{amssymb}
\usepackage{xspace}
\newcommand{\BibTeX}{\textsc{BibTeX}\xspace} % example of use of xspace to define a BibTeX macro
\newcommand{\latex}{\LaTeX\xspace}

% Commented text
\usepackage{comment}

% Packages for source code listings
\usepackage{listings}		%% source code listings
\usepackage{newfloat}
\DeclareFloatingEnvironment{listing}
\lstset{basicstyle=\footnotesize\ttfamily, %
columns=fullflexible, %
numbers=left, %
xleftmargin=1.5em, %
framexleftmargin=1.5em, %
keywordstyle=\color{blue}, %
stringstyle=\color{PineGreen}, %
keepspaces=true}


\title{A Guide to Survey Assignments % The title of your survey
%ACA course 2017/18 - project R# - Survey on ... %% please mention project ID and project name 
}
\author{G. Agosta, S. Cherubin} % 
\date{\today}

\begin{document}

\maketitle

% Section heading. You can use \subsection and \subsubsection to provide further subdivisions
\section{Introduction}
% Label definition, you can now refer to the number of this section through \ref{sec:intro} if needed.
\label{sec:intro}
% File inclusion. The contents of the Introduction section are in file intro.tex
This file provides a brief overview on how to carry out a survey assignment, and how to report on it.

\subsection{Carrying out the survey} % a subsection heading
The main goal of the survey is to understand the state of the art on a given topic. 
To this end, you are provided with a starting point, usually a recent paper on the topic. 
However, you should read more related works.
To find the relevant works, you can use specialized search engines such as  \href{https://scholar.google.com}{Google Scholar}.
Furthermore, digital libraries such as the \href{https://dl.acm.org}{ACM Digital Library} or \href{https://ieeexplore.ieee.org}{IEEExplore} will be useful.
% Note the use of links.
Please notice that papers of the aforementioned digital libraries are granted for free when you access from the Politecnico di Milano wireless or wired network.
% Leaving a blank line will start a new paragraph.

First, read the starting point paper(s). 
From it, you can get relevant keywords, as well as references to related works (there is usually a small review of related literature in each article).
Then, look for works that cite, are cited by the starting point paper(s). You may also wanto to investigate works done by the same authors of the starting paper(s).
Extend the set of papers, until you run out of works to read or things start getting out of hand -- if there are too many works in the field, you may want to focus your reading on a specific subset. In this case, in Section~\ref{sec:motiv} clearly state the boundaries you are going to use, and include there a short list of the most relevant works that cover the original assignment, but that you are not going to cover in detail.

When you read a paper, write down a brief summary of its key contributions. 
An example of paper summary annotation is presented in the Appendix~\ref{sec:append} of this paper.
This will help later on in writing the report.
Also, you are welcome to note down relevant keywords, which may help in definining a classification or taxonomy of the works you have surveyed and to look for more relevant works.
The goal of the survey is to understand which are the main research lines, so a classification of the literature by the chosen approach, sub-problem tackled, or other relevant parameter is necessary.

\subsection{Writing the report}
When you find a paper, look for its \BibTeX bibliographic record. 
You will find those in the above-mentioned digital libraries, or, failing that, on \href{http://dblp.uni-trier.de/}{DBLP} or \href{https://scholar.google.com}{Google Scholar}.
See the \texttt{biblio.bib} file in this folder for an idea of how the \BibTeX files look like.
You will need to build such a file, in order to record the papers you have read.

Once you have your bibliography, an understanding of its classification, and all the summaries, you are ready to produce your report.
For this purpose, use the \latex typesetting system~\footnote{See \href{https://en.wikibooks.org/wiki/LaTeX}{https://en.wikibooks.org/wiki/LaTeX}}. % Note the command for footnotes
This document also serves as a skeleton for your report -- just replace the contents of each \texttt{.tex} file with your text.
Also, useful information on how to write in \latex is reported in the comments to the source files -- so read them!

In the introduction, you should report a summary of your findings -- write it last, summarizing the contents of Sections~\ref{sec:motiv} and~\ref{sec:taxon}.
For the remaining sections, read the contents of each of them for an idea of how to write them.

To produce the \texttt{pdf} document out of the \latex sources, run the commands listed in Listing~\ref{lst:latex}.

\begin{listing}
\begin{lstlisting}[language=Bash]
pdflatex main.tex
bibtex main
pdflatex main.tex
pdflatex main.tex
\end{lstlisting}
\caption{\label{lst:latex}Commands to generate a \texttt{pdf} file from \latex sources. You need the first run of \texttt{pdflatex} to generate the bibliography references, then the last two to include them and regenerate the internal references in the \texttt{pdf} document.}
\end{listing}

\subsection{Writing the presentation}
Finally, in the presentation slides you should cover the same topics, focusing on the motivation and taxonomy. 
You may want to highlight the contents of a few of the most relevant or recent works as well, but do not go into too many details -- you only have 10/12 slides, of which one half should be spent on motivation, taxonomy, and future directions/conclusions.

You can use \latex for the slides as well -- using the document class \texttt{beamer}~\footnote{See \href{https://en.wikibooks.org/wiki/LaTeX/Presentations}{https://en.wikibooks.org/wiki/LaTeX/Presentations}} -- but this is not required.


% Note that & is a special character. Escape it with \ to use in text.
\section{Motivation \& Problem Statement}
\label{sec:motiv}
In this section, briefly summarize what is the problem addressed by the works you are surveying. Why are they useful? What problem(s) are they trying to solve? 


\section{Taxonomy/Feature Comparison}
\label{sec:taxon}
In this section, lay out the main findings of your reading assignement.

Essentially, you have to classify the works you read according to appropriate criteria. 

For example, if you were doing a survey of instruction scheduling techniques, you would consider as the major distinction whether the scheduling is performed in hardware or in software. 

Within software techniques, you would further classify based on the scope of the scheduling procedure (basic block, a loop, or a generic region).

You may want to provide a table, such as Table~\ref{tab:taxon}.
% Note the reference and the corresponding label

\begin{table}
\centering
\caption{\label{tab:taxon}A Taxonomy Table}
% "lclp" herebelow refers to the alignment: left, center, left, and wrap text at 0.5 the size of the page
\begin{tabular}{lclp{0.5\textwidth}}
\toprule
Reference & HW/SW & Scope & Other key features \\ % \\ ends a line, & acts as field separator.
\midrule
\cite{Thornton:1964:POC:1464039.1464045} & HW & Fixed window & Tomasulo's algorithm \\
\cite{Fisher:1981:TST:1311075.1311325} & SW & Global & Original trace scheduling algorithm\\
\bottomrule
\end{tabular}
\end{table}

Note that you can easily generate a \latex table from Excel or LibreOffice by exporting to CSV (Comma Separated Values) format, then loading the file with a simple Python script, as shown in Listing~\ref{lst:tables}.

\begin{listing}
\begin{lstlisting}[language=Python]
#!/usr/bin/python 
# load argument list and pandas library
from sys import argv
import pandas as pd
# load data from csv file. Filename is the last argument
data = pd.read_csv(argv[-1])
# convert data to latex, selecting only the columns you need
text = data.to_latex(
	index=False,
	columns=['column headings', 'you want to appear', 'in the document'])
# write latex text to a file
with open('tab.tex', 'w') as f:
	f.write(text)
\end{lstlisting}
\caption{\label{lst:tables}Code to generate a \latex table from a \texttt{.csv} file.}
\end{listing}


\section{Summary of relevant works}
\label{sec:rwork}
In this section, provide a brief summary (5-10 lines) of each surveyed work. 
Try to stress the key features that differentiate each work from the most obvious solution to the problem addressed.
You can organize the discussion following the timeline (i.e., listing the works by data), or by thematic groups (this is generally preferable if there clear taxonomical groups emerged in Section~\ref{sec:taxon}.
% Note the use of \ref{sec:taxon} to refer to the section on taxonomy. You can use this syntax to refer to any numbered item (e.g., figures).

Herebelow is an example of a summary for one of the papers listed in Table~\ref{tab:taxon}.
You can use it as a guideline for your own entries.

In~\cite{Fisher:1981:TST:1311075.1311325}, the \emph{Trace Scheduling} techniques is introduced to allow instruction scheduling across control flow barriers imposed by branches.
The proposed technique treats the primary trace, i.e. the most frequently executed sequence of basic blocks in the region, as a single basic block for scheduling purposes.   
This provides an increase in performance of the scheduler, at the cost of decreased performance for the secondary traces.
It is worth noting that the technique may fail entirely when the scheduling of the primary trace produces unrecoverable errors in the secondary traces (e.g., when a register used in a secondary trace is overwritten by a non-reversible operation, and its original value is lost).


\section{Conclusions}
\label{sec:conc}
In this section, sum up the primary research directions in the field of your survey, and highlight gaps in the state of the art or future directions (i.e., what you would do if you had to provide an innovative contribution to the field).


\bibliographystyle{plain}
\bibliography{biblio}

\newpage
\appendix

\section{Example of Structured Notes about a Paper}
\label{sec:append}
% appendix
%\linespread{.95}
\subsection*{Authors and affiliation}
Name1 Surname1, Name2 Surname2, ...

University of Grandma's Pizza, University of Grandpa's Grappa

\subsection*{Title}
Working on very interesting stuff: a case study\footnote{\url{link/to/paper/on/the/ACM/IEEE/whatever/website}}

%\subsection*{Affiliations}

\subsection*{Conference / Journal and Year of Publication}
International Conference of Computing and Cuisine, 1985

\subsection*{Why this paper is interesting}
This paper highlights the benefits of the approach based on electric oven over the traditional oven to cook lasagna.
It also features a very good related works section. Yeah, cool but no more than a couple of lines to write here.

\subsection*{Approach classification and scope}
baking methodologies
The proposed solution applies only to lasagna and not to generic cuisine.

\subsection*{Technologies involved}
Electric oven and traditional oven.

\subsection*{Innovation}
The innovation of this approach consists in using the electric oven to precook pasta for lasagna before cooking the lasagna in the traditional oven.

\subsection*{Benchmarks and metrics}
Cooking benchmarks. In particular, lasagna benchmark.
Metrics based on how good lasagna tastes.

\subsection*{Comments}
It looks strange to me, as pasta for lasagna should be precooked in water and not in the oven. However, there is no reference to water-cooking in this paper.



\end{document}


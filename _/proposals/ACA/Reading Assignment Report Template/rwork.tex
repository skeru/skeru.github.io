In this section, provide a brief summary (5-10 lines) of each surveyed work. 
Try to stress the key features that differentiate each work from the most obvious solution to the problem addressed.
You can organize the discussion following the timeline (i.e., listing the works by data), or by thematic groups (this is generally preferable if there clear taxonomical groups emerged in Section~\ref{sec:taxon}.
% Note the use of \ref{sec:taxon} to refer to the section on taxonomy. You can use this syntax to refer to any numbered item (e.g., figures).

Herebelow is an example of a summary for one of the papers listed in Table~\ref{tab:taxon}.
You can use it as a guideline for your own entries.

In~\cite{Fisher:1981:TST:1311075.1311325}, the \emph{Trace Scheduling} techniques is introduced to allow instruction scheduling across control flow barriers imposed by branches.
The proposed technique treats the primary trace, i.e. the most frequently executed sequence of basic blocks in the region, as a single basic block for scheduling purposes.   
This provides an increase in performance of the scheduler, at the cost of decreased performance for the secondary traces.
It is worth noting that the technique may fail entirely when the scheduling of the primary trace produces unrecoverable errors in the secondary traces (e.g., when a register used in a secondary trace is overwritten by a non-reversible operation, and its original value is lost).
